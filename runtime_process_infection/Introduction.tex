Writing malware on Windows is a relatively easy task. The win32 API is very friendly
towards malware authors. Microsoft even provides a library called detours which makes
injecting DLLs into a process during runtime extremely easy. Detours, however, does
not provide stealth. Linux has a similar project called injectso. Injectso is a
non-stealthy solution for injecting shared objects into a process during runtime.

If a system administrator thinks a service running on a server has been exploited,
one thing the administrator can do is look at the \texttt{/proc/[pid]/maps} file and
look at what shared objects are loaded. As an example, let's say we're targeting the
Apache web server. We want to drop into a shell whenever the string
``\texttt{GET /shell HTTP/1.1}" is sent to Apache over the socket. If we were to
use injectso, we'd inject a shared object in such a way that the administrator
would see an entry that matches \texttt{/path/to/evil.so}. Obviously, the system's
administrator knows at that point the game is over. And so will you when you try
to get back in.

Therefore, we need a stealthy way to inject our shared object. Our requirements are:
\begin{enumerate}
\item{Inject shared objects}
\item{Hook dynamically-loaded functions}
\item{Stealth through anonymous injection}
\end{enumerate}

In the rest of this paper, we will discuss:
\begin{enumerate}
\item{Current techniques for storing and executing arbitrary shellcode}
\item{The Executable and Linkable Format (ELF)}
\item{The new technique for storing and executing arbitrary shellcode}
\item{Hooking dynamically-loaded functions}
\item{Future work and research}
\end{enumerate}
