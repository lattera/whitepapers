Many techniques exist for storing and executing arbitrary code. For decades, storing
code on the stack has been a known technique\footnote{See aleph1's famous Phrack article,
\href{http://www.phrack.com/issues.html?issue=49&id=14&mode=txt}{Smashing The Stack For Fun And Profit}}.
Being well known, this technique is very well protected against\footnote{Add footnote for non-exec stack}.
We can store our shellcode on the heap. Using the heap would be really simple since
we can easily expand it with the \texttt{brk} system call. This technique is
relatively old as well. Many systems are setting the heap to be non-executable.
With the stack and the heap being protected, the other option would be to store our
code where \$eip points to. This would be an ideal solution, except that we need
to hook dynamically-loaded functions. Meaning, by overwriting the code at \$eip, we
destroy any chance of the program using that spot in memory again unless we backup
and restore the original code. Restoring the code would remove our injection,
meaning our code can only run once. All current techniques (including injectso) use
this method.

Since our main goal is to let our code run multiple times, we need a place in which
our code can reside. We can can store code on the stack and heap, but we can't
guarantee execution. We can store code at \$eip, but we are limited to a
run-once scheme. We have an interesting dillema on our hands: we have no place to
put our code and guarantee repeated execution.

Spoiler alert: We will inject into a new anonymous memory mapping. This mapping will
hold our malicious code and will be marked as executable. The next few chapters
discuss how we go about forcing the victim process to create a new memory mapping.
